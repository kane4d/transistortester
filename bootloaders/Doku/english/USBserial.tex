\chapter{Various USB to serial converter with Linux}

\section*{}
The classic serial interface is less and less used today.
On older computers this interface with RS232 standard is more common.
However, the classic serial interface is not very practical anyway, because
the voltage levels used (about -12V and +12V) can not be used directly.
This voltage levels must be first converted back to +5V or +3.3V common with
microcontrollers. 
For purpose of programming microcontrollers the USB to serial converter are
more practical, because these provide the proper signal level and can
also provide the +5V or +3.3V supply for the microcontroller.
With the Linux operating system these USB devives are most detected without problems.
But problems can occur to access this devices with the normal user access rights.
Mostly the access nodes like /dev/ttyUSB1 are assigned to the group dialout.
Then you (the user) should belong to this dialout group.
You can be member of the group dialout with the command ''usermod -a -G dialout \%USER''.


\section{The CH340G and the CP2102 converter }
Checked have I a board with the label BTE13-005A,
at which a CH340G converter from QinHeng Elektronics is mounted.
The board has a mini switch to switch the VCC voltage to 3.3V or 5V and a 12 MHz crystal.
At one side there is a USB-A plug and on the other side a 6-pin strip with signals
GND, CTS, VCC, TXD, RXD and DTR.
One LED connected to VCC and another LED is also mounted at the board.
You can find a list of supported baud rates in the chinese datasheet of the CH340G chip.
The other board with the CP2102 converter from Silicon Laboratories Inc. has no label.
At one side of the board you will find also a USB-A plug and at the opposite side there
is also a 6-pin strip with the signals 3.3V, GND, +5V, TXD, RXD and DTR.
At the two other sides of the board you can add a 4-pin strip with the 
signals DCD, D3R, RTS and CTS as well as RST, R1, /SUS and SUS.
All signals required by the bootloader are present already at the mounted strip.
But the sequence of the signals is different for both boards.
You can find only a few parts additional to the CP2102 converter chip,
but a LED for RXD, TXD and power is present.
With my Linux-Mint 17.2 both convertes are detected without further ado.
You can see the folowing output of the lsusb command:
\begin{verbatim}
Bus 002 Device 093: ID 10c4:ea60 Cygnal Integrated Products, Inc. CP210x UART Bridge / myAVR mySmartUSB light
Bus 002 Device 076: ID 0403:6001 Future Technology Devices International, Ltd FT232 Serial (UART) IC
\end{verbatim}
Of course, the information on the bus depends on the computer and the USB port used.
The automatically created device name after the connection can be find out by
the command ''dmesg | tail -20''. In the following example I have
summarized both results. 
\begin{verbatim}
usb 2-4.2: new full-speed USB device number 94 using ohci-pci
usb 2-4.2: New USB device found, idVendor=1a86, idProduct=7523
usb 2-4.2: New USB device strings: Mfr=0, Product=2, SerialNumber=0
usb 2-4.2: Product: USB2.0-Serial
ch341 2-4.2:1.0: ch341-uart converter detected
usb 2-4.2: ch341-uart converter now attached to ttyUSB1
usb 2-4.5: new full-speed USB device number 93 using ohci-pci
usb 2-4.5: New USB device found, idVendor=10c4, idProduct=ea60
usb 2-4.5: New USB device strings: Mfr=1, Product=2, SerialNumber=3
usb 2-4.5: Product: CP2102 USB to UART Bridge Controller
usb 2-4.5: Manufacturer: Silicon Labs
usb 2-4.5: SerialNumber: 0001
cp210x 2-4.5:1.0: cp210x converter detected
usb 2-4.5: reset full-speed USB device number 93 using ohci-pci
usb 2-4.5: cp210x converter now attached to ttyUSB2
\end{verbatim}

With this information and my own experiments I have created the tables~\ref{tab:CH340baudl} and~\ref{tab:CH340baudh}.
Because the newer operating system Linux Mint 18.3 has given partially better results, I have
measured the results with this system.

\begin{table}[H]
  \begin{center}
    \begin{tabular}{| c | c | c || c | c | c || c |}
    \hline
    CH340G     & CH340G & CH340G   &  CP2102   & CP2102 & CP2102    & AVR \\
    supported  & stty  & measured  & supported & stty   & measured  & UBBR  \\
    BaudRate   & speed & BaudRate  & BaudRate  & speed  & BaudRate  & @16MHz \\
    \hline
    \hline
         50    & 50    &   50.00   &    -      & Error  &           &       \\
    \hline
         75    & 75    &   75.18   &    -      & Error  &           &        \\
    \hline
        100    & Error &    -      &    -      & Error  &           &        \\
    \hline
        110    & 110   &   109.3   &    -      & Error  &           &        \\
    \hline
        134    & 134   &   133.4   &    -      & Error  &           &       \\
    \hline
        150    & 150   &   150.4   &    -      & Error  &           &        \\
    \hline
        300    & 300   &   300.7   &   300     &  300   &   300.7    &       \\
    \hline
        600    & 600   &   602.4   &   600     &  600   &   598.8   &  3332  \\
    \hline
        900    & Error &    -      &  (900)   &  Error &    -      &  2221 \\
    \hline
       1200    & 1200  &   1204.8  &   1200    &  1200  &   1198    &   832  \\
    \hline
       1800    & 1800  &   1801.6  &   1800    &  1800  &   1802    &   555 \\
    \hline
       2400    & 2400  &   2409.6  &   2400    &  2400  &   2410    &   416  \\
    \hline
       3600    & Error &   -       &  (3600)  &  Error &    -      &   277  \\
    \hline
      (4000)   & Error &   -       &   4000    &  Error &    -      &   249  \\
    \hline
       4800    & 4800  &   4808    &   4800    &  4800  &   4808    &   207  \\
    \hline
      (7200)   & Error &   -       &   7200    &  Error &    -      &   138  \\
    \hline
       9600    & 9600  &   9616    &   9600    &  9600  &   9616    &   207  \\
    \hline
      14400    & Error &   -       &  14400    & Error  &    -      &   138  \\
    \hline
        -      & Error &   -       &  16000    & Error  &    -      &   124  \\
    \hline
      19200    & 19200  &  19232   &   19200   &  19200 &  19232    &   103  \\
    \hline
    \end{tabular}
  \end{center}
  \caption{Tested baud rates of the CH340 and CP2102 chips at lower baud speed}
  \label{tab:CH340baudl}
\end{table}
 



\begin{table}[H]
  \begin{center}
    \begin{tabular}{| c | c | c || c | c | c || c |}
    \hline
    CH340G     & CH340G & CH340G    &  CP2102   & CP2102 & CP2102    & AVR \\
    supported  & stty   & measured  & supported & stty   & measured  & UBBR  \\
    BaudRate   & speed  & BaudRate  & BaudRate  & speed  & BaudRate  & @16MHz \\
    \hline
    \hline
      28800  &  Error   &   -       &  28800    & Error  &    -      &   68 \\
    \hline
      33600  &  Error   &   -       & (33600)   & Error  &    -      &   59  \\
    \hline
      38400  &  38400   &  38.464k  &  38400    & 38400  &  38.464k   &   51  \\
    \hline
     (51200) &  Error   &   -       &  51200    & Error  &    -      &   38,  0.16\%  \\
    \hline
      56000  &  Error   &   -       &  56000    & Error  &    -      &   35, -0.79\%  \\
    \hline
      57600  &  57600   &  57.8k    &  57600    & 57600  &  57.472k   &   34, -0.79\%  \\
    \hline
     (64000) &  Error   &   -       &  64000    & Error  &    -      &   30,  0.80\%  \\
    \hline
      76800  &  Error   &   -       &  76800    & Error  &    -      &   25, 0.16\%  \\
    \hline
     115200  &  115200  &  115.6k   &  115200   & 115200 & 114.96k   &   16, 2.12\%  \\
    \hline
     128000  &  Error   &   -       &  128000   & Error  &    -      &   15, -2.34\%  \\
    \hline
     153600  &  Error   &   -       &  153600   & Error  &    -      &   12, 0.16\%  \\
    \hline
     230400  &  230400  &  229.9k   &  230400   & 230400 &  229.9k   &   8, -3.54\%  \\
    \hline
    (25000)  &  Error   &   -       &  250000   & Error  &    -      &   7, 0.00\%  \\
    \hline
    (25600)  &  Error   &   -       &  256000   & Error  &    -      &   7, -2.34\%  \\
    \hline
     460800  &  460800  &  460.8k   &  460800   & 460800 & 458.7k    &   -, >5\%  \\
    \hline
    (500000) &  500000  &  500.0k   &  500000   & 500000 & 500.0k    &   3, 0.00\%  \\
    \hline
    (576000) &  576000 & \bf{543.4k} &  576000   & 576000 & 571.4k   &   -, >5\%  \\
    \hline
     921600  &  921600  &  \bf{851.2k} & 921600 & 921600 & 921.6k    &   -, >5\%  \\
    \hline
    (1000000) & 1000000  &  1000k    & (1000000) & 1000000 & \bf{921.6k}  &   1, 0.00\%  \\
    \hline
    (1200000) &  Error   &   -       & (1200000) & Error   &   -       &  -, >5\%  \\
    \hline
    1500000  & 1500000  &  1498k    & (1500000) & 1500000 & 1498k   &  -, >5\%  \\
    \hline
    2000000  & 2000000  &  2000k    & (2000000) & 2000000 & \bf{1504k}    & 0, 0.00\%  \\
    \hline
   (3000000) & 3000000   & 3007k    & (3000000) & 2000000 &   -       &  -, >5\%   \\
    \hline
    \end{tabular}
  \end{center}
  \caption{Tested baud rates of the CH340 and CP2102 chips at higher baud speed}
  \label{tab:CH340baudh}
\end{table}

In both tables you can see, that not all baud rates specified by the manufactorers
are adjustable with the Linux stty command.
In most cases an error is reported, but unfortunately not always.
With the CP2102 converter the error limit is exceeded at the baud rates 1 MBaud and 2 MBaud.
Even at the 576 kBaud rate the deviation is higher than expected.
With the CH340G converter there are only two baud rate noticeable, the setting 576000 and 921600.
All other baud selections have a uncritical tolerance.
I have not searched for the reason of this abnormalities.
But something is dubious with the documentation of the CH340G chip. 
We can select the baud rate 500 kBaud and 1 MBaud, which should be impossible
according to the documentation.


\section{The PL-2303 and the FT232R converter}
This time I have tested a board with the label ''SBT5329'' and the PL-2303RX converter from Profilic Technology
and a board ''FTDI Basic 1'' with a FT232RL converter from Future Technology Devices.

The SBT5329 board has a USB-A plug at one side and a 5-pin strip with the signals
+5V, GND, RXD, TXD, and 3.3V at the opposite side.
Additionally you can find a 12 MHz crystal and three LEDs Tx, Rx and power.
The control signals of the serial interface are not routed to the strip at this board.
But the PL-2303 chip provide the handshake signals of the serial interface.

The FTDI Basis 1 board has a USB-B plug at one side and at the opposite side a
6-pin strip with the signals GND, CTS, 5V, TXD, RXD and DTR. 
Only a few parts can be found at the boardadditional to the FT232RL chip.
But two LEDs for Tx and Rx are also present.

As with the other two boards, the devices are properly recognized by Linux Mint 17.1:
\begin{verbatim}
Bus 002 Device 095: ID 067b:2303 Prolific Technology, Inc. PL2303 Serial Port
Bus 002 Device 076: ID 0403:6001 Future Technology Devices International, Ltd FT232 Serial (UART) IC
\end{verbatim}
So the devices are shown with the command lsusb.
With the command ''dmesg | tail -20'' you can get direcly after plug in a
simular result as:
\begin{verbatim}
usb 2-4.5: new full-speed USB device number 95 using ohci-pci
usb 2-4.5: New USB device found, idVendor=067b, idProduct=2303
usb 2-4.5: New USB device strings: Mfr=1, Product=2, SerialNumber=0
usb 2-4.5: Product: USB-Serial Controller
usb 2-4.5: Manufacturer: Prolific Technology Inc.
pl2303 2-4.5:1.0: pl2303 converter detected
usb 2-4.5: pl2303 converter now attached to ttyUSB1
\end{verbatim}
or rather
\begin{verbatim}
usb 2-4.3: new full-speed USB device number 96 using ohci-pci
usb 2-4.3: New USB device found, idVendor=0403, idProduct=6001
usb 2-4.3: New USB device strings: Mfr=1, Product=2, SerialNumber=3
usb 2-4.3: Product: FT232R USB UART
usb 2-4.3: Manufacturer: FTDI
usb 2-4.3: SerialNumber: A50285BI
ftdi_sio 2-4.3:1.0: FTDI USB Serial Device converter detected
usb 2-4.3: Detected FT232RL
usb 2-4.3: Number of endpoints 2
usb 2-4.3: Endpoint 1 MaxPacketSize 64
usb 2-4.3: Endpoint 2 MaxPacketSize 64
usb 2-4.3: Setting MaxPacketSize 64
usb 2-4.3: FTDI USB Serial Device converter now attached to ttyUSB0
\end{verbatim}

In the table~\ref{tab:PL2302baudl} the baud setting for the FT232R converter 
is tested below 300 baud, which can not be solved by the chip.
But the stty command accept these settings without an error message.
Of course the resulting baud rate is incorrect for these settings.

\begin{table}[H]
  \begin{center}
    \begin{tabular}{| c | c | c || c | c | c || c |}
    \hline
    PL2303     & PL2303 & PL2303   &  FT232R   & FT232R & FT232R    & AVR \\
    supported  & stty  & measured  & supported & stty   & measured  & UBBR  \\
    BaudRate   & speed & BaudRate  & BaudRate  & speed  & BaudRate  & @16MHz \\
    \hline
    \hline
         75    &  75   &  75.18    &  (75)     &   75   &  \bf{415}  &        \\
    \hline
      (110)    &  110  &  109.9    &  (110)    & 110    & \bf{278} &        \\
    \hline
      (134)    &  134  &  135.1    &  (134)    & 134    & \bf{502}   &       \\
    \hline
        150    & 150   &  149.8    &  (150)    & 150    & \bf{832} &        \\
    \hline
        300    & 300   &  300.7    &   300     & 300    & 300.72     &       \\
    \hline
        600    & 600   &  602.4    &   600     & 600    & 602.4     &  3332  \\
    \hline
      (900)    & Error &    -      &   900     & Error  &    -      &  2221 \\
    \hline
       1200    & 1200  &  1204.8   &   1200    & 1200   & 1212    &   832  \\
    \hline
       1800    & 1800  &  1801.6   &   1800    & 1800   & 1809.6    &   555 \\
    \hline
       2400    & 2400  &  2409.6   &   2400    & 2400   & 2424    &   416  \\
    \hline
       3600    & Error &   -       &   3600    & Error  &    -      &   277  \\
    \hline
       4800    & 4800  &  4808     &   4800    & 4800   & 4831      &   207  \\
    \hline
       7200    & Error &   -       &   7200    &  Error &    -      &   138  \\
    \hline
       9600    & 9600  &  9616     &   9600    &  9600  &  9664     &   207  \\
    \hline
      14400    & Error &   -       &  14400    & Error  &    -      &   138  \\
    \hline
      19200    & 19200  & 19232    &  19200    & 19200  & 19320     &   103  \\
    \hline
    \end{tabular}
  \end{center}
  \caption{Tested baud rates of the PL-2303 and FT232R chips at lower baud speed}
  \label{tab:PL2302baudl}
\end{table}

In the table~\ref{tab:PL2302baudh} the baud setting for the FT232R converter 
are done correctly, when the stty command report no error.
I don't know the reason, why some baud rates are not accepted by the stty command
for the FT232R chip.
Only the 576000 setting can be solved more exactly by the chip than done here.

\begin{table}[H]
  \begin{center}
    \begin{tabular}{| c | c | c || c | c | c || c |}
    \hline
    PL2303     & PL2303 & PL2303   &  FT232R   & FT232R & FT232R    & AVR \\
    supported  & stty  & measured  & supported & stty   & measured  & UBBR  \\
    BaudRate   & speed & BaudRate  & BaudRate  & speed  & BaudRate  & @16MHz \\
    \hline
    \hline
    \hline
      28800  &  Error   &   -       &  28800    & Error  &    -      &   68 \\
    \hline
     (33600) &  Error   &   -       &  33600    & Error  &    -      &   59  \\
    \hline
      38400  &  38400   &  38.464k  &  38400    & 38400  &  38.6k   &   51  \\
    \hline
    (51200)  &  Error   &   -       &  51200    & Error  &    -      &   38,  0.16\%  \\
    \hline
    (56000)  &  Error   &   -       &  56000    & Error  &    -      &   35, -0.79\%  \\
    \hline
      57600  &  57600   &  57.8k    &  57600    & 57600  &  57.8k    &   34, -0.79\%  \\
    \hline
    (64000)  &  Error   &   -       &  64000    & Error  &    -      &   30,  0.80\%  \\
    \hline
    (76800)  &  Error   &   -       &  76800    & Error  &    -      &   25, 0.16\%  \\
    \hline
     115200  &  115200  &  115.6k   &  115200   & 115200 &  115.6k   &   16, 2.12\%  \\
    \hline
    (128000) &  Error   &   -       &  128000   & Error  &    -      &   15, -2.34\%  \\
    \hline
    (153600) &  Error   &   -       &  153600   & Error  &    -      &   12, 0.16\%  \\
    \hline
     230400  &  230400  &  231.2k   &  230400   & 230400 &  231.2k   &   8, -3.54\%  \\
    \hline
    (250000) &  Error   &   -       &  250000   & Error  &    -      &   7, 0.00\%  \\
    \hline
    (256000) &  Error   &   -       &  256000   & Error  &    -      &   7, -2.34\%  \\
    \hline
     460800  &  460800  &  460.8k   &  460800   & 460800  & 465.1k   &   -, >5\%  \\
    \hline
    (500000) &  500000  &  500.0k   &  500000   & 500000 & 500.0k    &   3, 0.00\%  \\
    \hline
    (576000) &  Error  &    -       &  576000 & 576000 & \bf{588.24k} &   -, >5\%  \\
    \hline
     921600  &  921600  &  925.6k   & 921600   & 921600 &  930.4k   &   -, >5\%  \\
    \hline
   (1000000) & 1000000 &   1000k    &  1000000 & 1000000 & 1005k    &   1, 0.00\%  \\
    \hline
   (1200000)  &  Error &   -        & 1200000  & Error   &   -      &  -, >5\%  \\
    \hline
   (1500000) & Error  &    1482k    & 1500000  & 1500000 &  1509k   &  -, >5\%  \\
    \hline
   (2000000) & 2000000 &   2010k   & 2000000 & 2000000 &  2020k   & 0, 0.00\%  \\
    \hline
   (3000000) & 3000000 &   3007k   & 3000000  & 3000000 &  3007k    &  -, >5\%   \\
    \hline
    \end{tabular}
  \end{center}
  \caption{Tested baud rates of the PL-2303 and FT232R chips at higher baud speed}
  \label{tab:PL2302baudh}
\end{table}


\section{The USB-serial converter with the ATmega16X2 software}

At some Arduino UNO boards a ATmega16X2 is used as USB-serial converter.
Therefore I would like to test this board for selectable baud rates.
At the first table~\ref{tab:mega16xbaudl} for the lower baud speeds is remarkable,
that baud rates below 600 are accepted by the stty command without error messages.
But all baud rates above 600 are implemented correctly.

\begin{table}[H]
  \begin{center}
    \begin{tabular}{| c | c | c || c |}
    \hline
    Mega16X2   & Mega16X2 & Mega16X2   & AVR \\
    supported  & stty  & measured  & UBBR  \\
    BaudRate   & speed & BaudRate  & @16MHz \\
    \hline
    \hline
         75    &  75   & \bf{956} &         \\
    \hline
        110    &  110  & \bf{1120}  &         \\
    \hline
        134    &  134  & \bf{757.6} &        \\
    \hline
        150    & 150   & \bf{1914} &         \\
    \hline
        300    & 300   & \bf{778}  &        \\
    \hline
        600    & 600   &  599     &   3332  \\
    \hline
        900    & Error &   -      &   2221 \\
    \hline
       1200    & 1200  &  1198    &    832  \\
    \hline
       1800    & 1800  &  1802    &    555 \\
    \hline
       2400    & 2400  &  2395    &    416  \\
    \hline
       3600    & Error &    -     &    277  \\
    \hline
       4800    & 4800  &  4808    &    207  \\
    \hline
       7200    & Error &    -     &    138  \\
    \hline
       9600    & 9600  &   9616   &    207  \\
    \hline
      14400    & Error &    -     &    138  \\
    \hline
      19200    & 19200  &  19232  &    103  \\
    \hline
    \end{tabular}
  \end{center}
  \caption{Tested baud rates of the ATmega16X2 at lower baud speed}
  \label{tab:mega16xbaudl}
\end{table}

At the table~\ref{tab:mega16xbaudh} for the upper baud speeds you can see,
that not all settings without error message are implemented correctly.
Of course a better feedback of the stty command would be desirable
to let us know, if a baud rate is selectable or not.

\begin{table}[H]
  \begin{center}
    \begin{tabular}{| c | c | c || c |}
    \hline
    Mega16X2  & Mega16X2 & Mega16X2   & AVR \\
    supported  & stty  & measured  & UBBR  \\
    BaudRate   & speed & BaudRate  & @16MHz \\
    \hline
    \hline
    \hline
      28800  &  Error   &   -       &    68 \\
    \hline
      33600  &  Error   &   -       &    59  \\
    \hline
      38400  &  38400   &  38.321k  &    51  \\
    \hline
      51200  &  Error   &   -       &    38,  0.16\%  \\
    \hline
      56000  &  Error   &   -       &    35, -0.79\%  \\
    \hline
      57600  &  57600   &  58.82k    &    34, -0.79\%  \\
    \hline
      64000  &  Error   &   -       &    30,  0.80\%  \\
    \hline
      76800  &  Error   &   -       &    25, 0.16\%  \\
    \hline
     115200  &  115200  &  116.9k   &    16, 2.12\%  \\
    \hline
     128000  &  Error   &   -       &    15, -2.34\%  \\
    \hline
     153600  &  Error   &   -       &    12, 0.16\%  \\
    \hline
     230400  &  230400  &  221.0k   &    8, -3.54\%  \\
    \hline
     250000  &  Error   &   -       &    7, 0.00\%  \\
    \hline
     256000  &  Error   &   -       &    7, -2.34\%  \\
    \hline
    (460800) &  460800  & \bf{500.0k} &    -, >5\%  \\
    \hline
     500000  &  500000  &  500.0k   &    3, 0.00\%  \\
    \hline
    (576000)  & 576000 & \bf{667k}  &    -, >5\%  \\
    \hline
    (921600) &  921600  & \bf{995.0k} &    -, >5\%  \\
    \hline
   (1000000) & 1000000 &   1000k    &    1, 0.00\%  \\
    \hline
   (1200000) &  Error   &   -        &   -, >5\%  \\
    \hline
   (1500000) & Error  & \bf{2000k} &   -, >5\%  \\
    \hline
    2000000  & 2000000 &   2000k   & 0, 0.00\%  \\
    \hline
   (3000000) & 3000000 & \bf{2000k} &   -, >5\%   \\
    \hline
    \end{tabular}
  \end{center}
  \caption{Tested baud rates of the ATmega16X2 at higher baud speed}
  \label{tab:mega16xbaudh}
\end{table}

